\documentclass{article}
\usepackage[final]{neurips_2020}

\usepackage[utf8]{inputenc}
\usepackage{ctex}
\usepackage[T1]{fontenc}    % use 8-bit T1 fonts
\usepackage{hyperref}       % hyperlinks
\usepackage{url}            % simple URL typesetting
\usepackage{booktabs}       % professional-quality tables
\usepackage{amsfonts}       % blackboard math symbols
\usepackage{amsmath}
\usepackage{nicefrac}       % compact symbols for 1/2, etc.
\usepackage{microtype}      % microtypography
\usepackage{indentfirst}
\usepackage{listings}
\usepackage{graphicx}
\usepackage{graphics}
\usepackage{float}
\usepackage[dvipsnames]{xcolor}
\definecolor{codegreen}{rgb}{0,0.6,0}
\definecolor{codegray}{rgb}{0.5,0.5,0.5}
\definecolor{codepurple}{rgb}{0.58,0,0.82}
\definecolor{backcolour}{rgb}{0.95,0.95,0.92}

\lstdefinestyle{mystyle}{
    backgroundcolor=\color{backcolour},   
    commentstyle=\color{codegreen},
    keywordstyle=\color{magenta},
    numberstyle=\tiny\color{codegray},
    stringstyle=\color{codepurple},
    basicstyle=\footnotesize,
    breakatwhitespace=false,         
    breaklines=true,                 
    captionpos=b,                    
    keepspaces=true,                 
    numbers=left,                    
    numbersep=5pt,                  
    showspaces=false,                
    showstringspaces=false,
    showtabs=false,                  
    tabsize=2
}

\lstset{style=mystyle}
\title{LAB0 设计文档}

\author{%
    21307099 李英骏\\
    \texttt{liyj323@mail2.sysu.edu.cn} \\
}

\begin{document}
\maketitle
\tableofcontents

\section{概述}
个人所得税计算器是一个简单的控制台应用程序,
用于根据用户的收入计算其应缴纳的个人所得税。
应用允许用户输入其月收入额,并基于预定义的税率和税级计算税款。
用户还可以调整税率和起征点以查看不同设置下的税款。

\section{类设计}

\subsection{Main}
\textbf{描述:}应用的入口点,负责初始化应用程序和启动用户界面。\\
\textbf{属性:}无\\
\textbf{方法:}
\begin{itemize}
    \item \texttt{main(String[] args)}:程序入口方法。
\end{itemize}

\subsection{TaxRate}
\textbf{描述:}存储和管理税率信息的类。\\
\textbf{属性:}
\begin{itemize}
    \item \texttt{double threshold}:免税起征点。
    \item \texttt{double[] rates}:税率数组。
    \item \texttt{double[] brackets}:每个税级的收入上限。
\end{itemize}
\textbf{方法:}
\begin{itemize}
    \item \texttt{getThreshold()}:返回免税起征点。
    \item \texttt{setThreshold(double)}:设置免税起征点。
    \item \texttt{getRates()}:返回税率数组。
    \item \texttt{setRates(double[])}:设置税率数组。
    \item \texttt{getBrackets()}:返回税级上限数组。
    \item \texttt{setBrackets(double[])}:设置税级上限数组。
    \item \texttt{getTaxRateTable()}:返回当前税率表的字符串表示形式。
\end{itemize}

\subsection{PersonalIncomeTaxCalculator}
\textbf{描述:}负责根据输入的收入计算税款的类。\\
\textbf{属性:}
\begin{itemize}
    \item \texttt{TaxRate taxRate}:税率对象。
\end{itemize}
\textbf{方法:}
\begin{itemize}
    \item \texttt{calculateTax(double)}:计算并返回根据给定收入的税款。
\end{itemize}

\subsection{UserInterface}
\textbf{描述:}管理用户交互的类。\\
\textbf{属性:}
\begin{itemize}
    \item \texttt{Scanner scanner}:输入扫描器。
    \item \texttt{PersonalIncomeTaxCalculator calculator}:税款计算器对象。
\end{itemize}
\textbf{方法:}
\begin{itemize}
    \item \texttt{showMenu()}:显示用户菜单并处理用户选择。
    \item \texttt{calculateTax()}:提示用户输入收入并显示计算的税款。
    \item \texttt{adjustThreshold()}:允许用户调整免税起征点。
    \item \texttt{adjustRates()}:允许用户调整税率。
\end{itemize}

\end{document}
